\documentclass[pdftex,11pt,a4paper]{article}
\usepackage{../cs444}

\begin{document}

\titlemake{4}

This document outlines the design of a name resolution, type checking,
and other semantic analysis phases of the Joos 1W language and
implemented in our compiler. It also discusses some challenges faced
while implementing these phases.


\section{Design}

Various components of the class were designed by augmenting or
replacing the ASTs that were generated from parsing. We have one AST
for each compilation unit that we are compiling. For the first
assignment there was only ever one AST, but as of the second
assignment the compiler handles an arbitrary number of ASTs being
created and processed.

\subsection{Name Resolution}

The first step in name resolution was to parse each AST and create
entity objects for each type of declared entity. We define an entity
in the same way that the JLS does:

\begin{quote}
A declared entity is a \tbf{package}, \tbf{class} type,
\tbf{interface} type, member (class,interface, \tbf{field}, or
\tbf{method}) of a reference type, \tbf{parameter} (to a method,
constructor, or exception handler), or \tbf{local variable}.
\end{quote}

For the entities that Joos supports; \ttt{Package}, \ttt{Class},
\ttt{Interface}, \ttt{Field}, \ttt{Method}, \ttt{Parameter}, and
\ttt{LocalVariable}; we have created classes that make accessing the
metadata of the entity easier and more efficient than repeatedly
navigating an AST.


\subsubsection{Type Environment}

After initially parsing the AST into the entity models, we can build
the type environment by building the package hierarchy and linking
imported types and packages into the compilation units that request
them.

Creating the package hierarchy\ldots

Importing\ldots

Resolving type names




\subsubsection{Hierarchy Checking}

Linking declarations\ldots

Environment building\ldots


Linking identifiers\ldots



\section{Challenges}

This phase of the compiler implementation presented many
challenges, some of which led to interesting solutions.

AST transformations


\subsection{Hacks}

Lots and lots of hacks\ldots

AST transformations


\end{document}


%%% Local Variables:
%%% mode: latex
%%% TeX-master: t
%%% End:
