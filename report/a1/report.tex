\documentclass[pdftex,11pt,a4paper]{article}
\usepackage{../cs444}

\begin{document}

\titlemake{1}

\section{Design}

Some fluffy preamble to design talk goes here.

As per assignment requirements, we implement scanning and parsing of
the Joos 1W language as specified by the Java Language Specification
2.0 (JLS) and the course website
\url{https://www.student.cs.uwaterloo.ca/~cs444/joos.html}.

Code is written in Ruby, and our implementation of the scanner and
lexer is located in \texttt{lib/}. Our copy of the Marmoset tests are
located in \texttt{test/}, and our own tests are located in
\texttt{spec/}. Offline tasks, such as running our parser generator
are located in \texttt{rakelib/}, which is the Ruby equivalent to the
\texttt{make} system. As a result, our \texttt{Makefile} delegates any
work to \texttt{rake} tasks.

The design of the scanning and parsing phases of our compiler are
broken up into five sections: the herp for derp, the foo for bar, the
baz for quux, hurr for durr, and AST for the final data structure that
we construct in this portion of the course.


\subsection{Tokens}
part of the weeding process from the lexer and parser.

Some more description. Maybe a code listing or example.


\subsection{Lexer}

It does stuff with a DFA and generates a stream of tokens from code
input.

Rejecting non-ASCII code is handled during this phase.


\subsection{Weeder}

Magic! Reject all programs that use non-Joos keywords, literal
integers that are outside of allowed ranges, and strings that include
invalid escape sequences.


\subsection{Abstract Syntax Tree}

Or whatever data abstraction we end up with for parsing.


\subsection{Parser}

It takes a stream of tokens from the lexer and does some other stuff.


\subsection{Other Requirements}

Some assignment requirements did not fit well into any other portion
of the compiler that has been designed so far. Or maybe not\ldots


\end{document}


%%% Local Variables:
%%% mode: latex
%%% TeX-master: t
%%% End:
