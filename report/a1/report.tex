\documentclass[pdftex,11pt,a4paper]{article}
\usepackage{../cs444}

\begin{document}

\titlemake{1}

This document outlines the design of a scanner and parser for
the Joos 1W language as specified by the Java Language Specification
2.0 (JLS) and the course website
\url{https://www.student.cs.uwaterloo.ca/~cs444/joos.html}. It also
discusses some challenges faced while implementing the scanner and
parser, as well as our strategy for testing correctness.

Code is written in Ruby, and our implementation of the scanner and
lexer is located in \ttt{lib/}. A copy of the Marmoset tests is
located in \ttt{test/}, while our own tests are located in
\ttt{spec/}. Offline tasks, such as running our parser generator
are located in \ttt{rakelib/}. All tasks that might have been in
the \ttt{Makefile} are delegate to tasks defined in
\ttt{rakelib/}.


\section{Design}

The design of the scanning and parsing phases of our compiler are
broken up into five sections: a description of our token data
structures which are produced by the parser and consumed by the
scanner. The scanner, which consumes input source code and produces
tokens. The abstract syntax tree (AST), which is produced by the
parser and will be used in later assignments. And finally, the parser,
which consumes tokens and produces the AST.


\subsection{Tokens}

The interface between the lexer and the parser is a stream
(\ttt{Array}) of \ttt{Joos::Token} objects. \ttt{Joos::Token}
is an abstract base class which encodes the original value of the
token, the type of the token, and metadata related to where the token
originated.

Concrete subclasses are created for each keyword, operator, separator,
and literal type, as defined in the JLS. A concrete class for
identifiers is also created in the \ttt{Joos::Token}
namespace. Classes may \ttt{include} modifiers which attach extra
metadata to a token.

The concrete token class itself specifies the type of the
token. For instance, a token representing the keyword \ttt{while} will
be contained by an instance of the \ttt{Joos::Token::While} class, and
a token representing a litertal \ttt{true} will be contained by an
instance of the \ttt{Joos::Token::True} class. Identifiers are wrapped
by \ttt{Joos::Token::Identifier}, literal integers by
\ttt{Joos::Token::Integer}, and so on.

Each token will contain metadata that can be used for diagnostic
purposes including, but not limited to, the relative path to the file
where the token comes from, the line in the file where the token comes
from, and the column in the line where the token begins. Additional
metadata is attached to different groupings of token classes to create
a sort of hierarchy. Additional metadata include things like the
\ttt{IllegalToken} modifier which is used to mark the types of tokens
that are legal Java tokens, but are not used in Joos.

Finally, concrete token classes will validate token values if
needed. In the case of string and character literals, we check that
all escape sequences in the string are legal escape sequences. Token
classes which are marked as illegal will raise an exception in their
constructor, which will cause the compiler to print an appropriate
error message and then exit.


\subsection{Lexer}

It does stuff with a DFA and generates a stream of tokens from code
input. Rejecting non-ASCII code is handled during this phase.


\subsection{Abstract Syntax Tree}

Our goal with the abstract syntax tree is to provide a data structure
that will make future stages of the compiler easier to implement.


\subsection{Parser}

It takes a stream of tokens from the lexer and does some other stuff.


\section{Challenges}


This is where the interesting shit goes. Talk about anything that did
not follow simply from class. This is a required section of the report.


\subsection{Weeder}

Weeding is performed during both scanning and parsing. Our goal is to
weed out issues as early as possible in order to make later stages
easier to implement, and so most weeding is done during scanning.

All keywords and operators not supported by Joos are caught during
scanning and will never be passed to the parser. Literal floating
point values and unsupported formats of literal integers are also
caught during the scanning phase.

However, validating that literal integers are within the signed 32-bit
range cannot be done during scanning because the range of values
differs depending on if the integer has a negative or positive value,
and we cannot determine that information until the tokens have been
parsed.


\subsection{Duplicate String Literals}

Literal string duplication is avoided by overriding how
\ttt{Joos::Token::String} objects are allocated. We override the
default allocation method for the class and check if another string
with the same binary value already exists. If it the string already
exists we return the existing token instead of a new token. The trade
off here is that we lose any metadata attached to the duplicate
literal string, such as the file and line where the string came from;
so any compilation errors related to the string may point to the
incorrect string, depending on the type of error. We compare against
the binary form of the string so that differences in escape sequences
do not affect the detection of duplicate strings.

I should probably modify this strategy a bit so that we actually keep
duplicate literal strings during the analysis phase of compilation,
and just link their in memory address when it comes to code
generation.


\section{Testing}

We wrote tests. And we also have an offline version of marmoset tests
which we run in the student environment at roflscale.

This section is required, so we have to fill it out at some point\ldots


\end{document}


%%% Local Variables:
%%% mode: latex
%%% TeX-master: t
%%% End:
