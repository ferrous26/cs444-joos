\documentclass[pdftex,11pt,a4paper]{article}
\usepackage{../cs444}

\begin{document}

\titlemake{5}

This document outlines the design of the code generation phase of
the Joos 1W language implemented by our compiler to target the
\ttt{i386} Linux platform.


\section{Design}

Ported runtime to OS X. Have both versions; choose the correct one at
compile time.

A code generator instance for each compilation unit.


\subsection{Generating Assembly Skeleton Files}

Symbol names, prefixing, etc\ldots

Literal strings

Static fields

What goes in the runtime library

Program startup


\subsection{Conventions}

Object layout

Arrays

Calling convention

Local function layout


\subsection{Substance}

Register allocator

SSA


\section{Testing}

Augment the test harness to compile stuff

Interestingly, tests that run extremely slowly are hidden by all the
tests which run quickly.

Augment the compiler so that output directory was configurable

Use our own exception routine in teh runtime so that we can print out
the excption type.

Add new tests to the marmoset suite to test edge cases in our
implementation.

Also added tests that exercise more components than the other tests.

In some cases, where test programs would not provide precise feedback,
we wrote unit tests.

In order to test the runtime, we create mock objects in assembly and
fed them to the various runtime functions to test that they worked
correctly.

\end{document}


%%% Local Variables:
%%% mode: latex
%%% TeX-master: t
%%% End:
